\section*{Long Abstract}
Uno dei problemi aperti della fisica moderna è quello riguardante la materia oscura. A supporto della sua esistenza vi sono innumerevoli osservazioni astronomiche, ma la sua natura fondamentale rimane ignota. Una delle più promettenti proposte per la spiegazione di tale fenomeno è rappresentata dalla possibile esistenza di ipotetiche particelle chiamate assioni.\\ %\cite{experimentalsearches} \cite{hunt}\\

L’idea degli assioni venne introdotta nel 1977 da Peccei e Quinn per risolvere il problema della mancata osservazione della violazione di CP nell'interazione forte tra quark. Queste particelle (insieme alle ALPs, Axion-Like Particles) dovrebbero interagire estremamente poco con le particelle del Modello Standard e perciò rappresentano dei validi candidati per la materia oscura.\\ 


Per la rivelazione degli assioni si sfrutta l’accoppiamento del campo assionico con il campo elettromagnetico (EM). Gli sforzi sperimentali vengono orientati sia alla ricerca di assioni di origine cosmica, sia in esperimenti condotti puramente in laboratorio.

%Per la rivelazione degli assioni si sfrutta l’accoppiamento del campo assionico con il campo Elettromagnetico (EM) e le ricerche sperimentali vengono essenzialmente condotte su due fronti: da un lato la ricerca di assioni di origine cosmica, dall'altro esperimenti condotti puramente in laboratorio.\\ 


%In particolare, la costante di accoppiamento $g_{a\gamma\gamma}$ è proporzionale alla massa a riposo dell’assione $m_ac^2$.
%Nel primo caso l’idea è quella di rivelare i fotoni prodotti dalla riconversione degli assioni generati da fonti astronomiche. 
Ad esempio, gli elioscopi servono a rivelare assioni prodotti nel nucleo solare, riconvertendoli in fotoni attraverso potenti magneti puntati verso il sole. Gli aloscopi invece hanno l'obiettivo di rivelare gli assioni che dovrebbero popolare l’alone galattico.\\
I principali esperimenti in laboratorio sono i “Light Shining through a Wall” (LSW), basati sulla generazione di assioni e sulla rigenerazione di fotoni tramite intensi campi magnetici. \\

In tutti questi casi, la rivelazione di assioni si basa sull’accoppiamento con i fotoni, tipicamente nelle microonde.  %Siccome tali particelle sono estremamente poco interagenti, la sfida sperimentale è quella di riuscire a rivelare singoli fotoni, .\\
Tali fotoni, nel caso di  LSW e aloscopi, vengono accumulati in cavità risonanti con frequenza di risonanza opportuna. Nelle condizioni sperimentali tipiche si stima un numero medio di fotoni pari a $\Bar{n}_{axion} \sim 10^{-8}-10^{-5}$ per misurazione \cite{PhysRevLett.126.141302}. La sfida sperimentale è dunque quella di riuscire a rivelare anche singoli fotoni nelle microonde.\\ Attualmente questi esperimenti utilizzano amplificatori lineari che operano al limite dello SQL (Standard Quantum Limit), le cui fluttuazioni corrispondono ad un background effettivo pari a $\Bar{n}_{SQL}=1$. 
Il rumore quantistico copre completamente il segnale ($\Bar{n}_{SQL}\gg\Bar{n}_{axion} $), rendendo impossibile la rivelazione.\\

Anziché utilizzare amplificatori lineari, si vogliono sviluppare dei rivelatori sensibili ad un singolo fotone. Esistono già delle tecnologie adatte a rivelare singoli fotoni nell’infrarosso, ma non sono altrettanto adatte a misurare fotoni di bassa energia nelle microonde.\\% come nel nostro caso.\\

È qui che entrano in gioco i transmoni, un tipo di qubit superconduttori costituiti da una giunzione Josephson e un condensatore di shunt.\\
Tali dispositivi sono alla base dei computer quantistici, ma giocano un ruolo chiave anche nella rivelazione di fotoni, sfruttando l’interazione
tra il qubit e il campo EM in una cavità risonante.

%sfruttando l'accoppiamento del con il campo EM in una cavità risonante.
%per rivelare i singoli fotoni si sfrutta l’interazione tra un qubit superconduttore e il campo EM nella cavità risonante. 
Infatti, la frequenza di transizione del qubit è legata al numero di fotoni nella cavità. Questo permette di sviluppare un protocollo di misura che possa inferire il numero di fotoni nella cavità a partire da una misura dello stato del qubit.\\

Il punto di forza di questo metodo è che si tratta di una misurazione "quantum non-demolition" ovvero tale da non perturbare il sistema quantistico che va ad indagare. Questo permette di superare lo SQL, dando un impulso notevole alla ricerca di particelle assioniche.% incrementando la velocità di ricerca di assioni di più ordini di grandezza.



